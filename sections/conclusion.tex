\section{Conclusiones}
Este trabajo consistió en la modelación matemática, así como la
construcción de una bocina direccional. Partiendo desde las ecuaciones
básicas de la acústica, y con base en un esquema de orden apropiado y
sólido, se construyó la ecuación de onda que describe los fenómenos de
difracción, absorción y no linealidad (la ecuación KZK). A partir de
ahí, los supuestos apropiados fueron adoptados para terminar con un
modelo que sirviera para simular el fenómeno y tener un marco de
referencia; pues la siguiente parte
del trabajo consistió en la experimentación sobre el fenómeno.\medskip\\
Como cierre para el trabajo, se exhiben los comentarios acerca de los
experimentos realizados en el capítulo anterior. ...
\section{Comment on your Results.} 
El trabajo consistió en el desarrollo de un modelo que explicara la
interacción no lineal en el campo cercano de las ondas ultrasónicas
(de AM), hasta el modelo KZK que describe de forma precisa la
propagación sonora que considera los efectos de difracción, absorción
y no linealidad del medio.
\subsection{Experimentos sobre la funcionalidad común del PA}
Para los primeros experimentos, se pudo observar que, a pesar de que
las mediciones no se hicieron en una cámara anecóica, éstas se
muestran bastante cercanas a las predicciones hechas en las
simulaciones. La figura ....
\subsection{Experimentos de cancelación sonora}
Estos experimentos, no pretendían establecer un sistema de control de
ruido como el mostrado en \cite{feasibility}, sino buscar algún
indicio de que existe algún tipo de interacción entre las ondas
ultrasónicas y las audibles. Los experimentos realizados en esta etapa
fueron mucho más sensibles al ruido externo, y las condiciones no
ideales fueron determinantes en el resultado, por lo menos de los
\section{Discussion}
